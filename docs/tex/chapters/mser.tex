\chapter{Maksymalnie stabilne regiony ekstremalne}

Maksymalnie Stabilne Regiony Ekstremalne (MSER) po raz pierwszy zostały opisane
w publikacji J. Matasa i innych pt. "Robust Wide Baseline Stereo from Maximally Stable
Extremal Regions" \cite{matas02} i znajdują zastosowanie np. w rozpoznawaniu i
śledzeniu obiektów, składaniu fragmentów obrazu oraz trójwymiarowej
rekonstrukcji otoczenia.

\section{Charakterystyczne regiony}

Charakterystyczne regiony (ang. distinguished regions) są fragmentami obrazów,
które posiadają określone, wyróżniające właściwości, pozwalające na ich
powtarzalne i trwałe wykrywanie, niezależnie od pewnych, często spotykanych,
deformacji. Maksymalnie stabilny region ekstremalny jest spójnym segmentem
obrazu otrzymanym w procesie binaryzacji przez progowanie. Właściwość
ekstremalności odnosi się do tego, że każdy piksel w danym rejonie jest
znacznie jaśniejszy (MSER+, lub ciemniejszy, MSER-) od wszystkich pikseli
otaczających region o parametr stabilności $\sigma$. Im jest on większy, tym
mniejsza liczba regionów spełni ten warunek. Zbyt mały próg może jednak
doprowadzić do zwrócenia wielu niepożądanych fragmentów obrazu, które
prawdopodobnie nie istnieją, lub istnieją w zupełnie innej formie, na kolejnym
obrazie w serii.  Ponieważ w trakcie jednego przebiegu algorytmu możemy szukać
tylko jednego typu regionów, wskazane jest powtórzenie algorytmu przetwarzacjąc
piksele w odwrotnej kolejności. Regiony te są natomiast maksymalnie stabilne
ponieważ w procesie binaryzacji ich rozmiar, w pewnym przedziale jasności, nie
rośnie.

Ten rodzaj detektora nie sprawdza się na powierzchniach mało kontrastowych
(takich jak niebo) oraz na teksturach, gdzie piksele o zbliżonej jasności nie
tworzą klastrów, gdyż są przemieszane z innymi, znacznie jaśniejszymi lub
ciemniejszymi pikselami.

MSER-y posiadają szereg pożadanych cech, w tym:

\begin{enumerate} \item niewrażliwość na zmianę perspektywy, \item obojętność
    na zmianę jasności (oświetlenia), \item liniową złożoność obliczeniową
    (choć w oryginalnej publikacji złożoność prawie liniowa), \item dużą
    dokładność opisywanych krawędzi (w przeciwieństwie do innych algorytmów
    stosujących smoothing czyli zmiękczanie krawędzi).  \end{enumerate}

Nawet po zajściu w/w transformacji, regiony, jeżeli tylko spełnią podstawowe
kryteria, będą dalej opisywać te same obszary.

\section{Algorytm}

Algorytm zaproponowany przez Davida Nistera i Henrika Steweniusa w publikacji
"Linear Time Maximally Stable Extremal Regions" \cite{Nister_Stewenius_2008}
jako pierwszy posiada, w najgorszym przypadku, liniową złożoność obliczeniową.
Najważniejszą zmianą w stosunku do wcześniejszych implementacji jest kolejność
w której iterujemy piksele. Dzięki temu struktura obszarów w pamięci
reprezentowana jest jako zbiór pikseli (a nie lista połączona). W analogii,
jasności pikseli tworzą pasma górskie, a kolejność odwiedzanych pikseli
determinuje woda (ze źródła znajdującego się w dowolnym miejscu), spływająca
najpierw do dolin tworząc jeziora, później unosząc się, łącząc kolejne
zbiorniki, w końcu zalewając cały obszar (w oryginalnej implementacji poziom
wody wzrasta jednomiernie na całej powierzchni obrazu, niezależnie od jego
topografii).

\subparagraph{Podstawowe definicje}

Załóżmy obraz \textit{I} posiadający \textit{n} pikseli indeksowanych zmienną
\textit{x}. Każdemu pikselowi przyporządkowana jest wartość \textit{f(x)}
będąca jasnością (ang. gray-scale value) z przedziału [0..255]. Każdemu
pikselowi odpowiada także zbiór \textit{N(x)} od 2 do 4 pikseli z nim
sąsiadujących. Ścieżka z \textit{x} do \textit{y} jest ciągiem sąsiadujących ze
sobą pikseli zaczynającym się w \textit{x} i kończącym się na \textit{y}.
Spójny zbiór \textit{X} jest zbiorem pikseli posiadającymi ścieżki każdy z
każdym. Zbiór z progiem $\Theta$ jest zbiorem pikseli których jasność nie
przekracza wartości $\Theta$. Wraz ze zwiększaniem progu \textit{y} (zwiększając
akceptowalną jasność pikseli) powstają nowe spójne zbiory, które tworzą drzewo
komponentów (ang. component tree). Liśćmi tego drzewa są tzw. minima, czyli
komponenty o pikselach o takiej samej jasności, a węzłami komponenty powstałe
po złączeniu dwóch innych komponentów, w wyniku procesu podnoszenia progu y do
momentu, w którym korzeniem drzewa stanie się komponent zawierający wszystkie
piksele obrazu.

W zależności od potrzeb, nie musimy pamiętać który piksel należy do którego
komponentu. Jeżeli w dalszym etapie analizy obrazu wystarczą nam przybliżone
położenie i kształt obszarów w postaci elips, to każdy region można opisać za
pomocą sum pikseli $(x,y)$: $\sum{1}, \sum{x}, \sum{y}, \sum{xx}, \sum{xy}, \sum{yy}$,
jednoznacznie opisujących elipsę. W dalszej części pracy zakładam, że
interesują nas wszystkie piksele należące do regionu, ponieważ jest to jedyny
sposób, aby na dalszym etapie uzyskać jego kontur.

Zbyt małe regiony stabilne nie są wystarczająco unikalne i z tego powodu nie są
interesujące, dlatego algorytm bierze pod uwagę tylko te regiony, których
rozmiar jest większy lub równy do parametru \textit{min}, określającego
wielkość w pikselach. Z drugiej strony, bardzo duży region ma małe szanse być
pomocny ze względu na większą wrażliwość na różne przekształcenia. Dodatkowo
mógłby on wpłynąć niekorzystnie na czas potrzebny do obliczenia deskryptora.
Parametr \textit{max} określa limit rozmiaru, po przekroczeniu którego nie są
sprawdzane warunki MSER (region nie może być wynikiem algorytmu).

\subparagraph{Struktury danych}

Implementacja wymaga następujących struktur danych: \begin{itemize} \item Maski
    binarnej odwiedzonych pikseli.  \item Kolejki priorytetowej granicznych
    pikseli (ang. boundary pixels) nienależących do sąsiadującego komponentu, z
    powodu wyższego poziomu jasności. Zostaną one zapamiętane od przetworzenia
    w dalszej kolejności, poczynając od najciemniejszego.  \item Stosu
    \textit{C} komponentów wraz z historią jego rozmiarów (niezbędą do
    stwierdzenia cechy stabilności), oraz poziomem jasności na którym aktualnie
    się znajduje.  \end{itemize}

\subparagraph{Przebieg algorytmu}

\begin{enumerate} \item Wyzeruj maskę binarną odwiedzonych pikseli, kolejkę
    piorytetową i stos komponentów. Włóż na stos pusty komponent z poziomem
    jasności większym niż jakikolwiek możliwy. \item Wybierz dowolny piksel
    źródłowy (np. lewy górny róg) i zapisz go do zmiennej
    \textit{current\_pixel}.  Zmień jego status w masce binarnej, a jego
    jasność zapisz do zmiennej \textit{current\_level}. \item Włóż na stos
    pusty komponent z poziomem jasności \textit{current\_level}. \item Odwiedź
    pozostałe piksele sąsiadujące. Dla każdego sąsiada sprawdź czy jest
    dostępny (tj. czy był odwiedzany) jeżeli nie jest, oznacz go jako dostępny
    i sprawdź czy jego jasność nie jest mniejsza od aktualnego poziomu, dodaj
    go do kolejki priorytetowej i przejdź do następnego kroku. W przeciwnym
    razie dodaj zawartość zmiennej \textit{current\_pixel} do kolejki, a
    zmiennej przypisz sąsiada. Uaktualnij wartość zmniennej
    \textit{current\_level}. Przejdź do punktu 3. \item Dodaj
    \textit{current\_pixel} do komponentu znajdującego się na wierzchu stosu
    komponentów. \item Pobierz piksel z kolejki priorytetowej.  Jeżeli w
    kolejce nie ma żadnych pikseli, zakończ algorytm. \item Jeżeli jasność
    zwróconego piksela jest równa $current\_level$, przejdź do punktu 4. W
    przeciwnym razie zwrócony piksel znajduje się wyżej (jest jaśniejszy) od
    przetwarzanego komponentu, istnieje możliwość połączenia dwóch lub więcej
    komponentów w jeden. W tym celu przetwarzamy wszystkie komponenty na
    stosie, których poziom jest mniejszy lub równy od obecnego. Dokonuje się
    tego za pomocą podprocedury \textit{ProcessStack}. Po jej wykonaniu wróć do
    punktu 4.  \end{enumerate}

Przebieg podprocedury \textit{ProcessStack(new\_pixel\_gray\_level)}
\begin{enumerate} \setcounter{enumi}{7} \item Rozważ komponent z wierzchołka
    stosu. Ustaw current\_level na mniejszą z wartości: new\_pixel\_gray\_level
    i poziomu drugiego komponentu na stosie. \item Jeżeli
    new\_pixel\_gray\_level jest mniejszy od poziomu drugiego komponentu na
    stosie, pozwól aby komponent na wierzchu stosu miał poziom
    new\_pixel\_gray\_level i wyjdź z procedury.  \item Zdejmij komponent z
    wierzchu stosu i połącz go z drugim komponentem na stosie z kolei.  Takie
    połączenie traktuje się jako pojedynczy skok rozmiaru większego z
    komponentów.  \item Jeżeli new\_pixel\_gray\_level jest większy od poziomu
    nowego komponentu na wierzchu stosu, przejdź do punktu 8.  \end{enumerate}

Po każdym wzroście poziomu komponentu o 1, należy sprawdzić czy spełnia on
kryteria MSER oraz czy jego rozmiar zawiera się w żądanym przedziale. Jeżeli
tak, zapamiętujemy do osobnej struktury, zawierającej wszystkie MSER-y, sam
kontur regionu.

\tikzstyle{decision} = [diamond, draw, fill=blue!20,
text width=2.5em, text badly centered, node distance=2cm, inner sep=0pt]
\tikzstyle{block} = [rectangle, draw, fill=blue!20,
text width=3em, text centered, rounded corners, minimum height=2em]
\tikzstyle{line} = [draw, -latex']
\tikzstyle{cloud} = [draw, ellipse,fill=red!20, node distance=3cm,
minimum height=2em]

\bigskip
\begin{figure}
\centering
\begin{tikzpicture}[node distance = 2cm, auto]
  % Place nodes
  \node [cloud] (start) {start};
  \node [block, right of=start] (1) {1};
  \node [block, right of=1] (2) {2};
  \node [block, right of=2] (3) {3};
  \node [decision, below of=3] (4) {4};
  \node [block, left of=4] (5) {5};
  \node [decision, left of=5] (6) {6};
  \node [cloud, left of=6] (stop) {stop};
  \node [decision, below of=6] (7) {7};
  \node [block, below of=7] (8) {8};
  \node [decision, right of=8] (9) {9};
  \node [block, right of=9] (10) {10};
  \node [decision, right of=10] (11) {11};
  % Draw edges
  \path [line, dashed] (start) -- (1);
  \path [line] (1) -- (2);
  \path [line] (2) -- (3);
  \path [line] (3.south) |- (4.north);
  \path [line] (4) -- node [near start] {} (5);
  \path [line] (4.east) -| node [near start] {} (3.east);
  \path [line] (5) -- (6);
  \path [line] (6) -- node [near start] {nie} (stop);
  \path [line] (6) -- node [near start] {tak} (7);
  \path [line] (7.east) -| node [near start] {tak} (4.south);
  \path [line] (7) -- node [near start] {nie} (8);
  \path [line] (9.north) -| node [near start] {tak} (4.south);
  \path [line] (9) -- node [near start] {nie} (10);
  \path [line] (10) -- (11);
  \path [line] (11) |- node [near start] {nie} (4.south);
  \path [line] (11.south) -| node [near start] {tak} (8.south);
  \path [line] (8) -- (9);
\end{tikzpicture}
  \caption{Diagram przepływu}
\end{figure}

\subparagraph{Implementacja}

Maskę binarną w języku JavaScript można przechowywać w zmiennej typu Integer,
która składa się z 32 bitów (tak naprawdę JavaScript nie rozróżnia liczb
całkowitych od zmiennoprzecinkowych, gdyż stosuje jeden standard IEEE 754, ale
niektóre operacje, w tym binarne, działają jedynie na pierwszych 32 bitach).
Aby obsłużyć większą liczbę pikseli (maska jednego pikslela to jeden bit),
tworzymy tablicę o rozmiarze n*m/32, gdzie n i m to długość i szerokość obrazu,
przechowującą 32 bitowe maski.

Kolejka priorytetowa to tablica 256 tablic pikseli o jasności odpowiadającej
indeksowi w pierwszej tablicy. Np. piątym elementem tablicy jest lista pikseli
czekających na przetworzenie, o jasności 5. Dodatkowo, w celu optymalizacji
czasu wykonania, wykorzystany został wskaźnik, pierwszego od końca, niepustego
elementu, do szybkiego dostępu do jego zawartości.

Stos komponentów jest tablicą 3-elementowych tablic, które zawierają:
\begin{inparaenum}[\itshape a\upshape)] \item aktualny poziom jasności
  komponentu, \item tablicę pikseli należących do komponentu, \item stos
  zawierający rozmiar komponentu na poprzednich poziomach jasności oraz \item
  rozmiar poprzedniego MSER, który pochodzi z danego regionu (jeżeli taki
  istnieje. Służy do wyeliminowania sytuacji w której zwracane są dwa, w dużej
  części nachodzące na siebie regiony).  \end{inparaenum}

Istnieje wiele możliwości wyznaczania jasności trójkolorowych pikseli.
Najszybsza metoda polega dodaniu do siebie wartości jednej czwartej części
czerwonego koloru oraz jednej drugiej części zielonego i niebieskiego.
