\chapter{Wstęp}

Eksponencjalny wzrost popularności urządzeń mobilnych obserwujemy już od wielu
lat.  Do tej pory jednak, mimo że wiele z nich wyposażone jest w kamerę
cyfrową, praktycznie żadne nie było w stanie przetwarzać obrazu nawet w czasie
zbliżonym do rzeczywistego. To zmieni się już niedługo, gdyż zgodnie z prawem
Moore'a moc obliczeniowa układów scalonych podwaja się co 24 miesiące.  Wielu
programistów nie chce jednak inwestować zbyt dużo energii na pisanie
skomplikowanych algorytmów na konkretną platformę sprzętową, gdyż udziały w
rynku nawet największych graczy, zmieniają się bardzo szybko i nie sposób
przewidzieć która z nich da największe korzyści w przyszłości. Rozwiązania
wieloplatformowe uznawane były do tej pory za zbyt wolne lub prymitywne.
Ostatnio jednak możemy zauważyć, że istnieje jedna technologia aktywnie
wspierana przez wszystkich producentów mobilnych systemów operacyjnych i jest
nią język programowania JavaScript oraz język znaczników HTML. Co więcej nowa
specyfikacja HTML5 dostarcza element canvas, który służy przetwarzaniu obrazów
na niskim poziomie i doskonale spełnia wymagania mojego projektu.

\section{Definicja problemu}

Rozpoznawanie i śledzenie obiektów to jeden z najbardziej znanych problemów
wizji komputerowej\footnote{Z języka angielskiego \textit{computer vision}.
Należy rozumieć jako komputerowa analiza obrazów w celu lepszego ich
zrozumienia.}. Mimo to temat ten ciągle budzi emocje, gdy nowe i lepsze
algorytmy zostają opublikowane. Rozwiązania dzielą się na dwa typy oparte na:
modelowaniu (ang. model-based) i analizie obrazu (ang. appearence-based).
Problemem tej pierwszej metody jest duży stopień skomplikowania i zawodność
podstawowych operacji takich jak wykrywanie krawędzi. Druga metoda zawodzi
natomiast, np.  przy zmianie kąta widzenia, gdy zmieniają się proporcje
wymiarów.  W roku 2002, został opublikowany nowatorski algorytm eliminujący
słabe punkty obu tych rozwiązań. J. Matas i inni, w \cite{matas02}
proponują łączenie lokalnych cech obrazu (ang. Local Affine Frames on
Distinguished Regions) i definiują pojęcie MSER (Maksymalnie stabilnych
regionów ekstremalnych), które posiadają wiele pożądanych cech brakujących we
wcześniejszych rozwiązaniach.  Kolejnym etapem analizy obrazu jest połączenie
charakterystycznych regionów (ang. distinguished regions) w pary. Najszybszym
obecnie algorytmem jest ISMatch opisany w "Efficient Partial Shape Matching of
Outer Contours" \cite{ismatch}, który dostarcza miarę podobieństwa dwóch
konturów. Dopiero teraz, mając wiele obszarów obrazu A, wraz z odpowiadającymi
im obszarami z obrazu B, możemy oszacować przekształcenia, które zaszły
pomiędzy tymi obrazami.

\section{Cel pracy}

Celem mojej pracy dyplomowej jest zaimplementowanie dwóch algorytmów: MSER i
ISMatch w języku JavaScript i zastosowanie ich do próby rozpoznania
przekształceń geometrycznych, które zaszły między dwoma obrazami, lub dwiema
klatkami filmu.  Przekształcenia afiniczne, które będą brane pod uwagę to:
zmiana położenia (przesunięcie), zmiana perspektywy (kąta widzenia), zmiana
kąta (obrót), zmiana wielkości obiektu (oddalenie lub przybliżenie).

\subparagraph{Maksymalnie stabilne regiony ekstremalne}: Algorytm MSER

W rozdziale 2 zostanie szczegółowo opisany zaimplementowany algorytm
znajdywania MSER-ów. Wyjaśnię jakie cechy musi spełniać zbiór pikseli obrazu,
aby można było je zakwalifikować do maksymalnie stabilnych ekstremów.  Podam
ich właściwości oraz warunki w których są one szczególnie korzystne oraz kiedy
zawodzą. Opiszę algorytm do znajdywania MSER-ów o liniowej złożoności
obliczeniowej.

\subparagraph{Dopasowanie kształtów}: Algorytm IS-Match

W rozdziale 3 zostanie szczegółowo opisany, częściowo zaimplementowany,
algorytm dopasowania konturów. Zacznę od zdefiniowania problemu i przeglądu
istniejących rozwiązań. Wyjaśnię dlaczego algorytm ISMatch jest najlepszy do
moich celów. Zdefiniuję wymaganą reprezentację kształtu i opiszę kątowy
deskryptor konturu, który nie dyskryminuje większości przekształceń
geometrycznych i pozwala na odnajdywanie częściowych dopasowań. Opiszę algorytm
dopasowań. Wyjaśnię dlaczego nie wszystkie cechy tego algorytmu są niezbędne do
mojej pracy i w jaki sposób można go uprościć zyskując na szybkości wykonania.

\subparagraph{Geometria epipolarna}

W rozdziale 4 zostanie zdefiniowany model geometrii epipolarnej. Przedstawię w
nim metody odtwarzania geometrii na podstawie układu dwóch zdjęć. Opiszę proces
budowy linii epipolarnych za pomocą macierzy fundamentalnej, oraz to w jaki
sposób możliwe jest wyeliminowanie błędnych sparowań regionów, powstałych we
wcześniejszej fazie przetwarzania. Wyjaśnię w jaki sposób, posiadając model
geometrii epipolarnej, można odtworzyć wszystkie przekształcenia afiniczne,
które zaszły między badanymi obrazami.

\subparagraph{Eksperymentalne wyniki i wydajność}

W rozdziale 5 dokonam pomiarów czasu wykonywania zaimplementowanych przeze mnie
algorytmów. Zbadam zależność przekazywanych parametrów na jakość i czas
pomiarów. Znajdę odpowiedź na pytanie czy obecna implementacja pozwala na
śledzenie obiektów w czasie rzeczywistym.

Uzyskane wyniki porównam do wyników innych implementacji publikowanych w
internecie.

\subparagraph{Wnioski i dalsza praca}

W rozdziale 6, podsumowującym, zaproponuję praktyczne zastosowanie mojej
implementacji oraz wykażę, że moja implementacja jest wartościowa i z
powodzeniem może zostać użyta w innych pracach jako gotowa biblioteka
programistyczna. Postaram się przewidzieć przyszłość podobnych rozwiązań i
zaproponuję udoskonalenia mojej implementacji.

\section{Powiązane prace i algorytmy}

Algorytm MSER jest częścią szerszego zagadnienia dotyczącego detektorów
lokalnych cech obrazów (ang. feature detection). Lokalna cecha obrazu jest
pewnym wykrojem, znacząco różniącym się od swojego otoczenia.  Najczęściej ma
to odzwierciedlenie w zmianie koloru, jasności lub tekstury.  Cecha może być
zarówno jednym punktem, krawędzią lub regionem (zbiorem połączonych pikseli).
Dzięki swoim unikalnym właściwościom, lokalne cechy obrazów były i są stałym
obiektem zainteresowania programistów i teoretyków, owocując licznymi
publikacjami i nowymi algorytmami, a przede wszystkim zastosowaniami
praktycznymi (w tym: sklejanie zdjęć satelitarnych, tworzenie panoram,
rekonstrukcja 3D, śledzenie obiektów, kalibracja itp.).

Tinne Tuytelaars i Krystian Mikolajczyk w "Local Invariant Feature Detectors:
A Survey" \cite{survey} rozróżniają trzy kategorie detektorów ze względu na
ich praktyczne wykorzystanie: \begin{enumerate}[a)] \item Wyszukiwanie
konkretnych typów cech ze względu na ich jasną interpretacje w danym
kontekście (np. krawędź - droga). \item Wyszukiwanie punktów zaczepnych (tzw.
zakotwiczeń). Wykorzystywane np. przy śledzeniu. \item Szybkie rozpoznawanie
scen, obiektów itp. na podstawie statystyk znalezionych cech. \end{enumerate} W
żadnym z tych przypadków nie jest tak na prawdę ważne która część obrazu
zostanie przez algorytm zakwalifikowana jako cecha, gdyż są one tylko środkiem
do uzyskania innego celu.

%\todo{porównanie cech algorytmów w tabeli}

\section{Dostępne implementacje}

W internecie dostępnych jest kilka bibliotek implementujących algorytmy
znajdujące lokalne cechy obrazów, wiele z nich jest publikowana wraz z kodem
źródłowym.

\subparagraph{VLFeat \cite{vlfeat}}

Otwarta biblioteka w języku C, implementuje wiele algorytmów do analizy obrazów
w tym MSER. Udostępnia programistyczny interface dla środowiska MATLAB.

\subparagraph{OpenCV \cite{opencv}}

Bardzo popularna biblioteka programistyczna napisana w językach C i C++
(posiada połączenia również do innych języków). Implementuje ponad 500
algorytmów z dziedziny wizji i grafiki komputerowej w tym MSER. Źródła są
publikowane na zasadzie wolnej licencji.

\subparagraph{Featurespace \cite{featurespace}}

Kod binarny programu napisanego przez Krystiana Mikolczyka, wykorzystywany przez
niego w wielu publikacjach. Oprócz MSER, implementuje również inne, podobne
algorytmy.
